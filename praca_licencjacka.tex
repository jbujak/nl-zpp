%
% Niniejszy plik stanowi przykład formatowania pracy magisterskiej na
% Wydziale MIM UW.  Szkielet użytych poleceń można wykorzystywać do
% woli, np. formatujac wlasna prace.
%
% Zawartosc merytoryczna stanowi oryginalnosiagniecie
% naukowosciowe Marcina Wolinskiego.  Wszelkie prawa zastrzeżone.
%
% Copyright (c) 2001 by Marcin Woliński <M.Wolinski@gust.org.pl>
% Poprawki spowodowane zmianami przepisów - Marcin Szczuka, 1.10.2004
% Poprawki spowodowane zmianami przepisow i ujednolicenie
% - Seweryn Karłowicz, 05.05.2006
% Dodanie wielu autorów i tłumaczenia na angielski - Kuba Pochrybniak, 29.11.2016

% dodaj opcję [licencjacka] dla pracy licencjackiej
% dodaj opcję [en] dla wersji angielskiej (mogą być obie: [licencjacka,en])
\documentclass[licencjacka]{pracamgr}


% Dane magistrantów:
\autori{Michał Borkowski}{370727}
\autorii{Jakub Bujak}{370737}
\autoriii{Marian Dziubiak}{370784}
\autoriv{Marek Puzyna}{371359}

\title{Kompilacja NianioLanga do efektywnych struktur języka C}


%\tytulang{An implementation of a difference blabalizer based on the theory of $\sigma$ -- $\rho$ phetors}

%kierunek:
% - matematyka, informacyka, ...
% - Mathematics, Computer Science, ...
\kierunek{informatyka}

% informatyka - nie okreslamy zakresu (opcja zakomentowana)
% matematyka - zakres moze pozostac nieokreslony,
% a jesli ma byc okreslony dla pracy mgr,
% to przyjmuje jedna z wartosci:
% {metod matematycznych w finansach}
% {metod matematycznych w ubezpieczeniach}
% {matematyki stosowanej}
% {nauczania matematyki}
% Dla pracy licencjackiej mamy natomiast
% mozliwosc wpisania takiej wartosci zakresu:
% {Jednoczesnych Studiow Ekonomiczno--Matematycznych}

% \zakres{Tu wpisac, jesli trzeba, jedna z opcji podanych wyzej}

% Praca wykonana pod kierunkiem:
% (podać tytuł/stopień imię i nazwisko opiekuna
% Instytut
% ew. Wydział ew. Uczelnia (jeżeli nie MIM UW))
\opiekun{mgr. Radosława Bartosiaka\\
  Instytut TODO\\
  }

% miesiąc i~rok:
\date{Maj 2018}

%Podać dziedzinę wg klasyfikacji Socrates-Erasmus:
\dziedzina{
%11.0 Matematyka, Informatyka:\\
%11.1 Matematyka\\
%11.2 Statystyka\\
11.3 Informatyka\\
%11.4 Sztuczna inteligencja\\
%11.5 Nauki aktuarialne\\
%11.9 Inne nauki matematyczne i informatyczne
}

%Klasyfikacja tematyczna wedlug AMS (matematyka) lub ACM (informatyka)
\klasyfikacja{D. Software\\
  D.3. Programming languages\\
  D.3.3. Language contructs and features}

% Słowa kluczowe:
\keywords{kompilacja, języki programowania, analiza semantyczna,
  NianioLang, system typów}

% Tu jest dobre miejsce na Twoje własne makra i~środowiska:
\newtheorem{defi}{Definicja}[section]

% koniec definicji

\begin{document}

\maketitle

%tu idzie streszczenie na strone poczatkowa
\begin{abstract}
  NianioLang jest językiem programowania ogólnego przeznaczenia, który można
  skompilować do wykonania na kilku platformach. Między innymi są to Java,
  JavaScript i C. Ze względu na chęć uproszczenia kompilacji NianioLanga
  utworzono środowisko uruchomieniowe przedstawiające odpowiednie abstrakcje,
  pozwalające na utworzenie dynamicznych struktur odpowiadających typom,
  jakie są dostępne do użycia w NianioLangu. Takie rozwiązanie nie jest
  niestety optymalne, szczególnie w przypadku niskopoziomowego języka jakim
  jest C. W tej pracy opisujemy wprowadzenie nowych typów i ich wsparcia
  w kompilatorze, które pozwolą na generowanie natywnego kodu w C, co znacznie
  zwiększy wydajność kompilowanych aplikacji.
\end{abstract}

\tableofcontents
%\listoffigures
%\listoftables

\chapter*{Wprowadzenie}
  \addcontentsline{toc}{chapter}{Wprowadzenie}
  \emph{O tym że jest to praca licencjacka, o tym kto nam to zadanie zlecił,
    itp.}

\chapter{Wstęp}
\section{Czym jest NianioLang?}
\section{Typy w NianioLangu}
\section{Cele projektu}

\chapter{Metodyka pracy}
\section{Korzystanie z systemu kontroli wersji}
\section{Zgłaszanie zmian i code review}
\section{Techniki komunikacji w zespole}

\chapter{Kompilator NianioLanga}
\section{Budowanie drzewa AST}
\section{Analiza semantyczna}
\section{Architektura nlasma}
  \emph{Rejestry, wywołania funkcji, deklaracje typów, itp.}
\section{Translacja drzewa AST do nlasma}
\section{Generowanie kodu C na podstawie nlasma}
\section{Implementacja typów NianioLanga w C}

\chapter{Rozszerzenie systemu typów}
\section{Rozdzielenie typu \texttt{ptd::sim}}
\section{Typy \texttt{own}}
  \emph{Jaki jest cel tych typów, ich semantyka, jakie ograniczenia na ich użycie
    nakładamy (w stosunku do typów ptd).}


\chapter{Rozszerzenie nlasma}
\section{Przekazywanie informacji o typach z drzewa AST}
\section{Statyczne sprawdzanie poprawności}

\chapter{Nowe implementacje typów}
  \emph{W tej sekcji w każdym podrozdziale będziemy opisywać dlaczego
    dotychczasowe rozwiązanie było nieefektywne, jak można je
    było poprawić, które rozwiązanie wybraliśmy, dlaczego.}
\section{Typy proste}
\section{Tablice}
\section{Rekordy}
\section{Typy wariantowe}

\chapter{Efekty optymalizacji i wnioski}
\section{Porównanie czasu wykonania programów}

\chapter{Wkład własny}
\emph{Co zrobiliśmy w rozbiciu na osoby}


\appendix

\begin{thebibliography}{99}
\addcontentsline{toc}{chapter}{Bibliografia}

\bibitem[Bea65]{beaman} Juliusz Beaman, \textit{Morbidity of the Jolly
    function}, Mathematica Absurdica, 117 (1965) 338--9.

\end{thebibliography}

\end{document}


%%% Local Variables:
%%% mode: latex
%%% TeX-master: t
%%% coding: latin-2
%%% End:
