\documentclass[11pt]{article}

\usepackage{polski}
\usepackage[polish]{babel}
\usepackage[utf8]{inputenc}

\setlength{\parindent}{0pt} % brak wcięć

\title{Optymalizacja kompilatora języka NianioLang do języka C}
\author{Jakub Bujak, Marek Puzyna, Marian Dziubiak, Michał Borkowski}
\date{2017/2018}

\begin{document}

    \maketitle

    \subsection*{Abstrakt}
    NianioLang jest językiem programowania ogólnego przeznaczenia, który można
    skompilować do wykonania na kilku platformach. Między innymi są to Java, 
    JavaScript i C. Ze względu na chęć uproszczenia kompilacji NianioLanga
    utworzono środowisko uruchomieniowe przedstawiające odpowiednie abstrakcje,
    pozwalające na utworzenie dynamicznych struktur odpowiadających typom,
    jakie są dostępne do użycia w NianioLangu. Takie rozwiązanie nie jest
    niestety optymalne, szczególnie w przypadku niskopoziomowego języka jakim
    jest C. W tej pracy opisujemy wprowadzenie nowych typów i ich wsparcia 
    w kompilatorze, które pozwolą na generowanie natywnego kodu w C, co znacznie
    zwiększy wydajność kompilowanych aplikacji.

    \tableofcontents

    \newpage

    \section{Wstęp}
    \emph{O tym że jest to praca licencjacka, o tym kto nam to zadanie zlecił,
          itp.}
    \subsection{Czym jest NianioLang?}
    \subsection{Typy w NianioLangu}
    \subsection{Cele projektu}

    \section{Metodyka pracy}
    \subsection{Korzystanie z systemu kontroli wersji}
    \subsection{Zgłaszanie zmian i code review}
    \subsection{Techniki komunikacji w zespole}

    \section{Kompilator NianioLang}
    \subsection{Budowanie drzewa AST}
    \subsection{Architektura nlasma}
    \emph{Rejestry, wywołania funkcji, deklaracje typów, itp.}
    \subsection{Translacja drzewa AST do nlasma}
    \subsection{Generowanie kodu C na podstawie nlasma}
    \subsection{Runtime NianioLang w języku C}
    \subsection{Implementacja typów NianioLangu w C}

    \section{Rozszerzenie systemu typów}
    \subsection{Rozdzielenie typu \texttt{ptd::sim}}
    \subsection{Typy \texttt{own}}
    \emph{Jaki jest cel tych typów, jakie są ograniczenia jakie na ich użycie
          nakładamy (w stosunku do typów ptd).}

    \section{Rozszerzenie nlasma}
    \subsection{Przekazywanie informacji o typach z drzewa AST}

    \section{Nowe implementacje typów}
    \emph{W tej sekcji w każdym podrozdziale będziemy opisywać dlaczego
          dotychczasowe rozwiązanie było nieefektywne, jak można je
          było poprawić, które rozwiązanie wybraliśmy, dlaczego.}
    \subsection{Typy proste}
    \subsection{Tablice}
    \subsection{Struktury}
    \subsection{Typy wariantowe}

    \section{Efekty optymalizacji i wnioski}
    \subsection{Porównanie czasu wykonania programów}

    \section{Wkład własny}
    \emph{Co zrobiliśmy w rozbiciu na osoby}

\end{document}
