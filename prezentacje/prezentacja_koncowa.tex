\documentclass{beamer}

\usepackage[utf8]{inputenc}
\usepackage{default}
\usepackage[OT4]{polski}
\usepackage[normalem]{ulem}
\usepackage{color}
\usepackage{graphicx}\usepackage{listings}
\usepackage{fancyvrb}
\usepackage{tikz}
\usepackage{listings}
\usepackage{hyperref}
\usetheme{Warsaw}
\graphicspath{ {files/} }

\tikzstyle{line} = [draw, -latex]
\tikzstyle{path} = [draw, -latex, style={decorate,decoration={snake}}]
\tikzstyle{state} = [circle,draw=black,minimum size=0.6cm]
\tikzstyle{finish_state} = [circle,draw=black,double=white,double distance=1pt,minimum size=0.9cm]
\usetikzlibrary{shapes,arrows,automata,decorations.pathmorphing}

\definecolor{dkgreen}{rgb}{0,0.6,0}
\definecolor{gray}{rgb}{0.5,0.5,0.5}
\definecolor{mauve}{rgb}{0.58,0,0.82}

\lstset{
  language=C,
  aboveskip=3mm,
  belowskip=3mm,
  showstringspaces=false,
  columns=flexible,
  basicstyle={\small\ttfamily},
  numbers=none,
  numberstyle=\tiny\color{gray},
  keywordstyle=\color{blue},
  commentstyle=\color{dkgreen},
  stringstyle=\color{mauve},
  breaklines=true,
  breakatwhitespace=true,
  tabsize=4,
  escapeinside=``
}

\lstdefinelanguage
	[x86fp]{Assembler}
	[x86masm]{Assembler} 
	{morekeywords={movss, xmm0, rip, rbp}}
   
\lstdefinelanguage
   [my]{C}
   []{C} 
   {morekeywords={bool}}
   
   
   \lstdefinelanguage{llvm}{
  morecomment = [l]{;},
  morestring=[b]", 
  sensitive = true,
  classoffset=0,
  morekeywords={
    define, declare, global, constant,
    internal, external, private,
    linkonce, linkonce_odr, weak, weak_odr, appending,
    common, extern_weak,
    thread_local, dllimport, dllexport,
    hidden, protected, default,
    except, deplibs,
    volatile, fastcc, coldcc, cc, ccc,
    x86_stdcallcc, x86_fastcallcc,
    ptx_kernel, ptx_device,
    signext, zeroext, inreg, sret, nounwind, noreturn,
    nocapture, byval, nest, readnone, readonly, noalias, uwtable,
    inlinehint, noinline, alwaysinline, optsize, ssp, sspreq,
    noredzone, noimplicitfloat, naked, alignstack,
    module, asm, align, tail, to,
    addrspace, section, alias, sideeffect, c, gc,
    target, datalayout, triple,
    blockaddress
  },
  classoffset=1, keywordstyle=\color{purple},
  morekeywords={
    fadd, sub, fsub, mul, fmul,
    sdiv, udiv, fdiv, srem, urem, frem,
    and, or, xor,
    icmp, fcmp,
    eq, ne, ugt, uge, ult, ule, sgt, sge, slt, sle,
    oeq, ogt, oge, olt, ole, one, ord, ueq, ugt, uge,
    ult, ule, une, uno,
    nuw, nsw, exact, inbounds,
    phi, call, select, shl, lshr, ashr, va_arg,
    trunc, zext, sext,
    fptrunc, fpext, fptoui, fptosi, uitofp, sitofp,
    ptrtoint, inttoptr, bitcast,
    ret, br, indirectbr, switch, invoke, unwind, unreachable,
    malloc, alloca, free, load, store, getelementptr,
    extractelement, insertelement, shufflevector,
    extractvalue, insertvalue,
  },
  alsoletter={\%},
  keywordsprefix={\%},
}


\lstdefinelanguage{nl}{
  morecomment = [l]{\#},
  morestring=[b]', 
  sensitive = true,
  classoffset=0,
  morekeywords={
    return, match, case, if, elsif, else, while, continue, try,
  },
  classoffset=1, keywordstyle=\color{purple},
  morekeywords={
	def, var,
  },
  alsoletter={:},
  keywordsprefix={},
}

\begin{document}
	\begin{frame}
		\frametitle{ZPP -- NianioLang}
		\center{Kompilacja NianioLanga do  efektywnych konstrukcji języka C}
		
		\center{15.06.2018}
	\end{frame}
	
	\begin{frame}
		\frametitle{NianioLang}
		\begin{itemize}
		 \item Język imperatywny bez wskaźników
		 \item Niemutowalne zmienne
		 \item Opcjonalne typowanie
		\end{itemize}
	\end{frame}
	
	\begin{frame}
		\frametitle{Cel projektu}
		Umożliwienie generowania efektywnego kodu wynikowego C
	\end{frame}
	
	\begin{frame}
		\frametitle{Nowe typy \texttt{own}}
		\begin{itemize}
		 \item Brak kopiowania
		 \item Jeden właściciel w dowolnej chwili
		 \item Zmienne kompilowane to prostych struktur języka C
		 \item Zmienne na stosie tam, gdzie to możliwe
		\end{itemize}
	\end{frame}
	
	\begin{frame}[fragile]
		\frametitle{Porównanie kodu wynikowego}
		\center{\texttt{a[0]->b++}}
			\pause
		\begin{columns}
			\begin{column}{0.50\textwidth}
				\lstset{language=C}
				\begin{lstlisting}
move(&nl_2,get_const(0));
move(&nl_1,get_arr(nl_0, nl_2));

move(&nl_3,get_const("b"));
move(&nl_3,get_hash(nl_1, nl_3));

move(&nl_4,get_const(1));
move(&nl_3,add(nl_3, nl_4));


move(&nl_5,get_const("b"));
set_hash(&nl_1,nl_5,nl_3);
set_arr(&nl_0,nl_2,nl_1);
				\end{lstlisting}
			\end{column}
			\pause
			\begin{column}{0.50\textwidth}
				\lstset{language=C}
				\begin{lstlisting}
nl_int_2 = 0;
nl_rec_ptr_1 = type0get_ptr(&nl_arr_0, nl_int_2);
nl_int_ptr_3 = &(nl_rec_ptr_1->b0field);


nl_int_4 = 1;
(*nl_int_ptr_3) = (*nl_int_ptr_3) + nl_int_4;
`\phantom{a}`
`\phantom{a}`
`\phantom{a}`
				\end{lstlisting}
			\end{column}
		\end{columns}
	\end{frame}
	
	\begin{frame}
		\frametitle{Efekty pracy}
		\begin{itemize}
			\item Przyspieszenie w stosunku do starych typów 10-100 razy
			\item Testy automatyczne dla nowych typów
			\item Skompilowanie kompilatora nim samym
		\end{itemize}

	\end{frame}
	
	\begin{frame}[c]
		\begin{center}
			\Huge \color{blue} Demo
		\end{center}
	\end{frame}
\end{document}
