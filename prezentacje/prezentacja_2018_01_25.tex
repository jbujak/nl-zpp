\documentclass{beamer}

\usepackage[utf8]{inputenc}
\usepackage{default}
\usepackage[OT4]{polski}
\usepackage[normalem]{ulem}
\usepackage{color}
\usepackage{graphicx}
\usepackage{listings}
\usepackage{fancyvrb}
\usetheme{Warsaw}

\begin{document}
	\begin{frame}
		\frametitle{ZPP -- NianioLang}
		Końcowosemestralna prezentacja
		
		25.01.2018
	\end{frame}
	
	\begin{frame}
		\frametitle{Przegląd ostatniego tygodnia}
		\begin{itemize}
			\item Dalsza praca nad kompilacją dostępów do zagnieżdżonych pól struktur
			-- Jakub Bujak, brak postępów
			\item Poprawki warningów kompilacji gcc -- Marian Dziubiak, zejście z 275 na 74 warningi
			\item Dostosowywanie type checkera do nowych typów -- Michał Borkowski, wymuszenie
			inicjalizacji typów own i zabronienie kopiowania
			\item Dodanie skryptu benchmarkującego -- Michał Borkowski, zrobione
		\end{itemize}
	\end{frame}
	
	\begin{frame}
		\frametitle{Co zostało zrobione w tym semestrze}
		\begin{itemize}
			\item Znacząca rozbudowa type checkera
			\item Dodanie kompilacji intów i booli NL do intów i booli C
			\item Dodanie generowania definicji struktur dla rekordów typu own
			\item Dodanie kompilowania inicjalizacji rekordów
		\end{itemize}
	\end{frame}
	
	\begin{frame}
		\frametitle{Rozbudowa type checkera}
		\begin{itemize}
			\item Dodanie zapisywania typów w drzewie AST
			\item Poprawki do wnioskowania o typie zmiennych
			\item Otypowanie brakujących funkcji bibliotecznych
			\item Dodanie zapisywania informacji o typie każdej analizowanej wartości
			\item Wymuszanie inicjalizacji ownów i zakaz kopiowania zmiennych own
		\end{itemize}
	\end{frame}
	
	\begin{frame}
		\frametitle{Dodanie kompilacji intów i booli}
		\begin{itemize}
			\item Otypowanie rejestrów nlasma
			\item Ustawianie typów rejestrów podczas generowania nlasma używając informacji z AST
			\item Dodanie generowania różnych konstrukcji C dla różnych typów rejestrów
		\end{itemize}
	\end{frame}

	\begin{frame}
		\frametitle{Generowanie definicji struktur dla rekordów typu own}
		\begin{itemize}
			\item Sortowanie topologiczne zdefiniowanych rekordów, tak żeby typy pól były wcześniej
			niż typ rekordu je zawierającego
			\item Wypisywanie definicji struktur w kolejności topologicznej
			\item Wypisywanie \texttt{typedef} dla każdego zdefiniowanego typu, tak żeby można było się
			do niego odnosić bezpośrednio po jego nazwie
			\item Na razie nazwy struktur są generowane tylko na podstawie nazw z NianioLanga
			-- psuje się przy anonimowych typach rekordów
		\end{itemize}
	\end{frame}

	\begin{frame}
		\frametitle{Dodanie kompilowania inicjalizacji rekordów}
		\begin{itemize}
			\item Wypisywanie ciągu deklaracji struktur pomocniczych i przypisań
			\item Niezmiennik: w nlasmie pomiędzy dowolnymi dwoma instrukcjami rekord jest w
			poprawnym stanie (nie ma częściowej inicjalizacji).
			\item Inicjalizacja może przebiegać w czasie kwadratowym od liczby pól w rekordzie
			(przy zagnieżdżonych strukturach kopiujemy całe te struktury)
		\end{itemize}
	\end{frame}
	
	\begin{frame}
		\frametitle{Co zostało do zrobienia}
		\begin{itemize}
			\item Dodanie kompilacji dostępów do pól rekordu
			\item Dodanie obsługi anonimowych typów rekordów
			\item Dodanie kompilacji pozostałych typów:
			\begin{itemize}
				\item tablica \texttt{own::arr(type)} do trójki
				$\langle\texttt{int size, int capacity, type *elements}\rangle$
				\item wariant \texttt{own::var} do pary
				$\langle\texttt{int label, void *element}\rangle$
				\item \texttt{own::string} do tablicy znaków
			\end{itemize}
		\end{itemize}
	\end{frame}

	\begin{frame}
	 \frametitle{Dziękuję za uwagę}
	\end{frame}

\end{document}
