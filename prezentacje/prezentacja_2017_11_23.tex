\documentclass{beamer}

\usepackage[utf8]{inputenc}
\usepackage{default}
\usepackage[OT4]{polski}
\usepackage[normalem]{ulem}
\usepackage{color}
\usepackage{graphicx}
\usepackage{listings}
\usepackage{fancyvrb}
\usetheme{Warsaw}

\begin{document}
	\begin{frame}
		\frametitle{ZPP -- NianioLang}
		Cotygodniowa prezentacja
		
		16.11.2017
	\end{frame}
	
	\begin{frame}
		\frametitle{Co miało być zrobione}
		\begin{itemize}
			\item Modyfikacja przydzielania rejestrów, tak żeby uniknąć niepoprawnych typów w kodzie C
			\item Dodanie indeksowania tablic intami
			\item Dokończenie zadań z zeszłego tygodnia:
			\begin{itemize}
				\item Integracja intów
				\item Implementacja przypisań na intach
				\item Implementacja operacji arytmetycznych na intach
			\end{itemize}
		\end{itemize}
	\end{frame}
	
	\begin{frame}[fragile]
		\frametitle{Modyfikacja rejestrów -- główny cel tygodnia}
		\begin{itemize}
			\item Bardzo duża zmiana
			\item Spowodowała ponad 400 błędów kompilacji
			\item A wygląda tak niewinnie:\newline
			\begin{Verbatim}[commandchars=\\\{\}]
def nlasm::reg_t() \{
\color{red}-       return ptd::sim();
\color{green}+       return ptd::var(\{
\color{green}+               im => ptd::sim(),
\color{green}+               int => ptd::sim(),
\color{green}+               string => ptd::sim(),
\color{green}+               bool => ptd::sim()
\color{green}+       \});
\}
			\end{Verbatim}
		\pause
		\item Tylko co się tutaj stało?
		\end{itemize}
	\end{frame}
	
	\begin{frame}[fragile]
		\frametitle{Rejestry nlasm}
		\textit{nlasm -- język pośredni podczas kompilacji NianioLanga}
		\begin{itemize}
		 \item Formalnie maszyna rejestrowa
		 \item Mamy do dyspozycji nieskończenie wiele rejestrów i podstawowe operacje na nich
		 \item Przykładowy program dla \texttt{var x = 11 + 22}:
		 \begin{Verbatim}
registers => [0, 1, 2]
cmds => [load_const(dest => 1, val => 11),
         load_const(dest => 2, val => 22),
         bin_op(dest => 0, left => 1,
                right => 2, op => '+')]
		 \end{Verbatim}
		\end{itemize}
	\end{frame}

	\begin{frame}[fragile]
		\frametitle{Przydzielanie rejestrów}
		\begin{itemize}
		\item Każda zmienna otrzymuje własny rejestr
		\item Kiedy zmienna wychodzi z zasięgu, jej rejestr jest zwalniany\pause
		\begin{Verbatim}[commandchars=\\\{\}]
\{                           \color{teal}#reg = 0\pause
	var a;              \color{teal}#reg = 1, a: 0\pause
	\{        
		var b;      \color{teal}#reg = 2, a: 0, b: 1\pause
	\}                   \color{teal}#reg = 1, a: 0\pause
	var c;              \color{teal}#reg = 2, a: 0, c: 1\pause
\}                           \color{teal}#reg = 0\pause
		\end{Verbatim}
		\item Problem pojawia się, kiedy zmienne \texttt{b} i \texttt{c} są różnych typów
		\end{itemize}
	\end{frame}
	
	\begin{frame}
		\frametitle{Rozwiązanie}
		\begin{itemize}
			\item \sout{Wszyskie rejestry \texttt{void*}}
			\item \textbf{Oddzielna pula rejestrów dla każdego typu}
			\item Wymaga dodania do rejestru informacji o typie
			\item Przepisanie mechanizmu przydzielania rejestrów
			\item Naprawienie wszystkich miejsc w kodzie, gdzie założono że rejestr jest liczbą
			\item Praca kiepsko się zrównolegla między członków zespołu
		\end{itemize}
	\end{frame}
	
	\begin{frame}[fragile]
		\frametitle{Efekt -- Przed}
		\begin{Verbatim}
void* ___nl__0 = NULL;
void* ___nl__1 = NULL;
#line 5 // var i = 11;
c_rt_lib0move(&___nl__1,___get_global_const(57));
c_rt_lib0copy(&___nl__0, ___nl__1);
c_rt_lib0clear(&___nl__1);

#line 6 // i = i + i;
c_rt_lib0move(&___nl__1, c_rt_lib0add(___nl__0, ___nl__0));
c_rt_lib0copy(&___nl__0, ___nl__1);
c_rt_lib0clear(&___nl__1);
c_rt_lib0clear(&___nl__0);
	\end{Verbatim}

	\end{frame}

	\begin{frame}[fragile]
		\frametitle{Efekt -- Po}
		\begin{Verbatim}
INT ___nl__int__0 = 0;
INT ___nl__int__1 = 0;
#line 5 // var i = 11;
___nl__int__1 = 11;
___nl__int__0 = ___nl__int__1;

#line 6 // i = i + i
___nl__int__1 = ___nl__int__0 + ___nl__int__0;
___nl__int__0 = ___nl__int__1;
	\end{Verbatim}

	\end{frame}

	\begin{frame}
		\frametitle{Plany na najbliższy tydzień}
		\begin{itemize}
		\item Nic
		\end{itemize}
	\end{frame}
\end{document}
