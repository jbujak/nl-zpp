\documentclass{beamer}

\usepackage[utf8]{inputenc}
\usepackage{default}
\usepackage[OT4]{polski}
\usepackage[normalem]{ulem}
\usepackage{color}
\usepackage{graphicx}
\usetheme{Warsaw}

\begin{document}
	\begin{frame}
		\frametitle{ZPP -- NianioLang}
		Cotygodniowa prezentacja
		
		16.11.2017
	\end{frame}
	
	\begin{frame}
		\frametitle{Co miało być zrobione}
		\begin{itemize}
		 \item Usunięcie \texttt{ptd::sim()}
		 \item Integracja intów
		 \item Implementacja przypisań na intach
		 \item Implementacja operacji arytmetycznych na intach
		\end{itemize}
	\end{frame}
	
	\begin{frame}
		\frametitle{Usunięcie \texttt{ptd::sim()}}
		\begin{itemize}
			\item Osoba odpowiedzialna -- Jakub Bujak
			\item\textcolor{green}{Zrobione}
			\begin{itemize}
				\item Wydzielenie intów z \texttt{ptd::sim()}
				\item Typowanie stałych liczbowych jako intów
				\item Poprawki do biblioteki standardowej -- konieczność utworzenia kopii na potrzeby kompilatora
				\item Przed tą zmianą instukcja \texttt{var a = '1' + 2} była poprawna
			\end{itemize}
			\item\textcolor{red}{Nie zrobione}
			\begin{itemize}
				\item Całkowite usunięcie \texttt{ptd::sim()} -- niemożliwe zanim nie dodamy stringów
			\end{itemize}

		\end{itemize}
	\end{frame}
	
	\begin{frame}
		\frametitle{Implementacja operatorów na intach}
		\begin{itemize}
			\item Osoba odpowiedzialna -- Michał Borkowski
			\item\textcolor{green}{Zrobione}
			\begin{itemize}
				\item Definicja operatorów działających na intach w bibliotece C
				\item Dodanie generowania nowych operatorów przez przez kompilator
				\item Dodanie obsługi intów w istniejącym systemie rejestrów
			\end{itemize}
			\item\textcolor{red}{Do zrobienia}
			\begin{itemize}
				\item Przy bardziej skompilowanych wyrażeniach generowany jest niepoprawny kod C
			\end{itemize}
		\end{itemize}
	\end{frame}
	
	\begin{frame}
		\frametitle{Implementacja operatorów na intach -- przykład}
		NianioLang:\newline
		\texttt{var i : ptd::int() = 1;}\newline
		\texttt{i = i + i;}\newline
		\newline
		C:\newline
		\texttt{INT nl1;}\newline
		\texttt{INT nl2;}\newline
		\texttt{move(nl2, const(59));}\newline
		\texttt{copy(nl1,nl2);}\newline
		\texttt{clear(nl2);}\newline
		\texttt{move(nl2, int0add(nl1,nl1));}\newline
		\texttt{copy(nl1, nl2);}\newline
	\end{frame}
	
	\begin{frame}
		\frametitle{Implementacja przypisań na intach}
		\begin{itemize}
			\item Osoba odpowiedzialna -- Marek Puzyna
			\item\textcolor{green}{Zrobione}
			\begin{itemize}
				\item Przypisanie stałej na inta
				\item \texttt{var a : ptd::int() = 1;}
			\end{itemize}
			\item\textcolor{red}{Do zrobienia}
			\begin{itemize}
				\item Przypisanie inta na inta
				\item \texttt{var a : ptd::int() = 1;}\newline
				\texttt{var b : ptd::int() = a;}
			\end{itemize}
		\end{itemize}
	\end{frame}

	\begin{frame}
		\frametitle{indeksowanie tablic intami}
		\begin{itemize}
			\item Osoba odpowiedzialna -- Marian Dziubiak
			\item\textcolor{yellow}{W trakcie}
		\end{itemize}
	\end{frame}
	
	\begin{frame}
		\frametitle{Plany na najbliższy tydzień}
		\begin{itemize}
			\item Modyfikacja przydzielania rejestrów, tak żeby uniknąć niepoprawnych typów w kodzie C
			\item Dokończenie zadań z tego tygodnia
			\item Dodanie indeksowania tablic intami
		\end{itemize}
	\end{frame}
\end{document}
