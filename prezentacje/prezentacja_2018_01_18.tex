\documentclass{beamer}

\usepackage[utf8]{inputenc}
\usepackage{default}
\usepackage[OT4]{polski}
\usepackage[normalem]{ulem}
\usepackage{color}
\usepackage{graphicx}
\usepackage{listings}
\usepackage{fancyvrb}
\usetheme{Warsaw}

\begin{document}
	\begin{frame}
		\frametitle{ZPP -- NianioLang}
		Poświąteczna prezentacja
		
		18.01.2018
	\end{frame}
	
	\begin{frame}
		\frametitle{Co miało być zrobione}
		\begin{itemize}
			\item Zdebugowanie kompilacji intów
			\item Doprowadzenie do wygenerowania kompilującego się kodu C
			\item Zapisywanie typów zmiennych w drzewie AST
			\item Początek pracy nad kompilacją struktur
			\item Wyjaśnienie obsługi sytuacji wyjątkowych używając async/await
		\end{itemize}
	\end{frame}
	
	\begin{frame}[fragile]
		\frametitle{Zdebugowanie kompilacji intów + generowanie poprawnego kodu C + typy w AST}
		\color{green} Zrobione
		\begin{itemize}
			\item Osoba odpowiedzialna -- Jakub Bujak
			\item Dodanie zapisywania wywnioskowanych typów zmiennych w drzewie AST
			\item Dodanie wnioskowania o typach wszystkich wartości pojawiających się w kodzie i ich zapisywanie
			\item Poprawienie generowania przypisań między rejestrami o różnych typach
		\end{itemize}
	\end{frame}
	
	\begin{frame}[fragile]
		\frametitle{Początek kompilacji struktur}
		\color{green} Zrobione
		\begin{itemize}
		 \item Osoby odpowiedzialne -- Jakub Bujak, Michał Borkowski
		 \item Dopracowanie generowania deklaracji struktur
		 \item Dodanie inicjalizacji struktur
		 \item Dodanie nowych warunków do type checkera
		 \item Pierwsze próby kompilacji odwołań do pól rekordu
		\end{itemize}
	\end{frame}
	
	\begin{frame}
		\frametitle{Wyjaśnienie obsługi sytuacji wyjątkowych używając async/await}
		\color{green} Zrobione
		\begin{itemize}
			\item Osoba odpowiedzialna -- Marek Puzyna
		\end{itemize}
	\end{frame}
	
	\begin{frame}
		\frametitle{Dodatkowo -- dodanie kompilacji booli}
		\color{green} Zrobione
		\begin{itemize}
			\item Osoba odpowiedzialna -- Marian Dziubiak
			\item Dzięki temu pusta pętla kompiluje się do w pełni efektywnego C
			\item Efekt: z 9s do 0.3s dla $10^8$ pustych iteracji (analogiczny for w C działa w 0.24s)
		\end{itemize}
	\end{frame}
	
	\begin{frame}
	 \frametitle{Dziękuję za uwagę}
	\end{frame}

\end{document}
