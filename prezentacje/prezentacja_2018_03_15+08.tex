\documentclass{beamer}

\usepackage[utf8]{inputenc}
\usepackage{default}
\usepackage[OT4]{polski}
\usepackage[normalem]{ulem}
\usepackage{color}
\usepackage{graphicx}
\usepackage{listings}
\usepackage{fancyvrb}
\usetheme{Warsaw}

\begin{document}
	\begin{frame}
		\frametitle{ZPP -- NianioLang}
		Krótka prezentacja
		
		08.03.2018
	\end{frame}
	
	\begin{frame}
		\frametitle{Co było do zrobienia}
		\begin{itemize}
			\item Ruszenie kompilacji tablic -- \textcolor{green}{ok}
			\item Przygotowania do dużej prezentacji -- \textcolor{green}{ok}
			\item Doprowadzenie do uruchomienia wszystkich testów -- \textcolor{red}{nie ok}
		\end{itemize}
	\end{frame}
	
	\begin{frame}
		\frametitle{Co zrobiliśmy}
		\begin{itemize}
			\item Prawie skończona kompilacja tablic -- Jakub Bujak
			\item Obsługa ownów w type checkerze i bugfixy -- Marek Puzyna, Michał Borkowski
			\begin{itemize}
				\item Zakaz zwracania ownów z funkcji
				\item Zakaz przekazywania ownów przez wartość
				\item Zakaz wielokrotnego przekazywania tego samego owna jako ref
			\end{itemize}
			\hrulefill
			\item Slajdy do dużej prezentacji -- Jakub Bujak, Marian Dziubiak
			\item Wykrywanie dostępów do niezainicjalizowanych zmiennych -- Marek Puzyna
			\item Pierwsza wersja konwersji own $\rightarrow$ im -- Michał Borkowski
			\item Dokończenie kompilacji tablic i testy do nich -- Jakub Bujak
			\item Przygotowanie spisu treści pracy licencjackiej -- Marian Dziubiak
		\end{itemize}
	\end{frame}
	
	\begin{frame}[fragile]
		\frametitle{Tablice own::arr}
		\begin{columns}
			\begin{column}{0.5\textwidth}
				\begin{Verbatim}[commandchars=\\\{\}]
def main::t() \{
 return own::arr(ptd::int());
\}

def main::main() \{
 var b : @main::t= [4, 2];
 b[1] = 42;
\}
				\end{Verbatim}
			\end{column}
			\begin{column}{0.5\textwidth}  %%<--- here
				\begin{Verbatim}[commandchars=\\\{\}]
struct main0t0type \{
 INT capacity;
 INT size;
 INT *value;
\};

main0t0type arr_0 = \{\};
main0t0push(&arr_0, 4);
main0t0push(&arr_0, 2);
int_ptr_4 = main0t0get_ptr(
  &arr_0, 1);
*int_ptr_4 = 42;
				\end{Verbatim}
			\end{column}
		\end{columns}
	\end{frame}
	
	\begin{frame}
		\frametitle{Plany na najbliższy tydzień}
		\begin{itemize}
			\item Dokońzenie konwersji own $\rightarrow$ im
			\item Ruszenie kompilacji stringów
			\item Stworzenie listy rzeczy, które zostały do zrobienia
			\item Początek pisania tekstu pracy licencjackiej
		\end{itemize}
	\end{frame}
\end{document}
