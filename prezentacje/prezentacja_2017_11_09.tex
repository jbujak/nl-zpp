\documentclass{beamer}

\usepackage[utf8]{inputenc}
\usepackage{default}
\usepackage[OT4]{polski}
\usepackage[normalem]{ulem}
\usepackage{color}
\usepackage{graphicx}
\usetheme{Warsaw}

\begin{document}
	\begin{frame}
		\frametitle{ZPP -- NianioLang}
		Cotygodniowa prezentacja
		
		09.11.2017
	\end{frame}
	
	\begin{frame}
		\frametitle{Co miało być zrobione}
		\begin{itemize}
		 \item Dokończyć konfigurację Travisa
		 \item Dodać kompilację intów NianioLanga do intów C
		 \item Dodać obsługę intów w type checkerze
		 \item Przygotować spis wymagań
		\end{itemize}
	\end{frame}
	
	\begin{frame}
		\frametitle{Sprecyzowanie wymagań}
		\begin{itemize}
			\item\textcolor{green}{Zrobione}
			\begin{itemize}
				\item Celem projektu jest dodanie do języka nowych typów w sposób umożliwiający kompilowanie
				intów, stringów, tablic i rekordów do odpowiednich typów C
				\item Po zmodyfikowaniu kompilatora należy przygotować benchmarki, które sprawdzą przyrost wydajności
				przy użyciu nowych typów
				\item Sprawdzenie poprawności wykonania ma nastąpić przez uruchomienie w środowisku produkcyjnym
				programu skompilowanego nowych kompilatorem przy zmianie typów części zmiennych na nowe
				\item Osoba odpowiedzialna -- Jakub Bujak
			\end{itemize}
		\end{itemize}
	\end{frame}
	
	\begin{frame}
		\frametitle{Dokończenie konfiguracji Travisa}
		\begin{itemize}
			\item\textcolor{green}{Zrobione}
			\begin{itemize}
				\item Projekt jest budowany i testowany przy każdym pull requeście
				\item Nie można merge'ować niebudującego się PR
				\item Osoba odpowiedzialna -- Marian Dziubiak
			\end{itemize}
		\end{itemize}
	\end{frame}
	
	\begin{frame}
		\frametitle{Kompilacja intów NianioLanga do intów C}
		\begin{itemize}
			\item\textcolor{green}{Zrobione}
			\begin{itemize}
				\item Przeniesienie pełnej informacji o typach do AST -- Jakub Bujak
				\item Generowanie deklaracji zmiennej w C -- Marek Puzyna
			\end{itemize}
			\item\textcolor{yellow}{Częściowo zrobione}
			\begin{itemize}
				\item Dodanie polecenia deklaracji zmiennej do nlasm -- Marian Dziubiak
			\end{itemize}
			\item\textcolor{red}{Do zrobienia}
			\begin{itemize}
				\item Integracja poszczególnych części i testy
			\end{itemize}
		\end{itemize}
	\end{frame}
	
	\begin{frame}
		\frametitle{Dodanie obsługi intów w type checkerze}
		\begin{itemize}
			\item\textcolor{green}{Zrobione}
			\begin{itemize}
				\item Możliwość przypisania inta tylko na inta (brak wsparcia dla starych typów prostych)
				\item Dzięki temu można będzie wyrzucić z języka poprzedni typ prosty -- \texttt{ptd::sim()}
				\item Osoba odpowiedzialna -- Michał Borkowski
			\end{itemize}
		\end{itemize}
	\end{frame}
	
	\begin{frame}
		\frametitle{Plany na najbliższy tydzień}
		\begin{itemize}
			\item Integracja i testowanie kompilacji intów
			\item Implementacja podstawowych operacji na intach -- odczyt/zapis i działania arytmetyczne
		\end{itemize}
	\end{frame}
\end{document}
