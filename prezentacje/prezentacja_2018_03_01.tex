\documentclass{beamer}

\usepackage[utf8]{inputenc}
\usepackage{default}
\usepackage[OT4]{polski}
\usepackage[normalem]{ulem}
\usepackage{color}
\usepackage{graphicx}
\usepackage{listings}
\usepackage{fancyvrb}
\usetheme{Warsaw}

\begin{document}
	\begin{frame}
		\frametitle{ZPP -- NianioLang}
		Prezentacja po feriach
		
		01.03.2018
	\end{frame}
	
	\begin{frame}
		\frametitle{Co zrobiliśmy przez ferie}
		\begin{itemize}
			\item Dokończenie dostępów do struktur, włącznie z zagnieżdżonymi -- Jakub Bujak
			\item Dalsze poprawki warningów kompilacji gcc -- Marian Dziubiak, Jakub Bujak, warningi:
			$74 \rightarrow 6$
			\item Dodanie nazewnictwa anonimowych struktur -- Michał Borkowski
			\item Przerobienie inicjalizacji na ciąg przypisań -- Jakub Bujak
			\item Poprawki do testów -- Jakub Bujak
		\end{itemize}
	\end{frame}
	
	\begin{frame}[fragile]
		\frametitle{Kolejne benchmarki}
		\begin{itemize}
			\item Przypomnienie: po kompilacji intów 9s $\rightarrow$ 0.3s
			\item Teraz jest jeszcze lepiej
			\begin{Verbatim}[commandchars=\\\{\}]
rep var i (10000000) \{
       a->b->c = 2;
       a->b->c++;
\}
			\end{Verbatim}
			\pause
			\item ptd::rec -- 71s
			\item own::rec -- 0.6s
		\end{itemize}

	\end{frame}

	
	\begin{frame}
		\frametitle{Plany na najbliższy tydzień}
		\begin{itemize}
			\item Ruszenie kompilacji tablic
			\item Doprowadzenie do uruchomienia wszystkich testów
			\item Duża prezentacja
		\end{itemize}
	\end{frame}

	\begin{frame}
	 \frametitle{Dziękuję za uwagę}
	\end{frame}

\end{document}
